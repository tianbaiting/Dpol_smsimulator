
\documentclass[aspectratio=169]{beamer}
\usetheme{Madrid}
\usecolortheme{default}

% Packages
\usepackage{graphicx}
\usepackage{amsmath}
\usepackage{amssymb}
\usepackage{booktabs}
\usepackage{listings}
\usepackage{xcolor}
\usepackage{hyperref}
\usepackage{tikz}
\usepackage{xfp}
\usepackage{subcaption}
% duplicate graphicx removed
\usepackage{pgffor} % [EN] Required for \foreach loop / [CN] 用于foreach循环
\usetikzlibrary{shapes,arrows,positioning}

% Code listing settings
\lstset{
  basicstyle=\ttfamily\footnotesize,
  breaklines=true,
  frame=single,
  numbers=left,
  numberstyle=\tiny\color{gray},
}
\title{dpol config}
\author{Tian Baiting}
\institute{SAMURAI Collaboration}

% Footer customization
\setbeamertemplate{footline}[frame number]

\begin{document}

% Title slide
\begin{frame}
\titlepage
\end{frame}

\begin{frame}
  %  list of content 生成目录
\tableofcontents  
\end{frame}
\section{config}

\subsection{rough scan}

\begin{frame}
    \frametitle{rough scan on parameter}

Angle : 2, 4, 6, 8, 10 degree

magnetic field: 0.8, 1.0, 1.2, 1.6, 2.0 T

    
    For magnetic field settings below 1.2 T, deuterons exit from the neutron exit window. Above 1.2 T, protons are prone to collisions. Therefore, a more detailed parameter scan test is conducted between 1.0 and 1.2 T at frame. see detailed in after  frame\ref{detailed}{click to jump to detailed scan}.

\end{frame}

% [EN] Combined trajectory visualization for D-beam test / [CN] D束流测试轨迹可视化

% [EN] Define parameter lists for magnetic field scan / [CN] 定义磁场扫描参数列表
\def\fields{080, 100, 120, 160, 200}

% [EN] Generate slides automatically using foreach loop / [CN] 使用foreach循环自动生成幻灯片
\foreach \b in \fields {
    \begin{frame}{Combined Trajectories: $B = \b \times$0.01 T}
        \centering
        \begin{tabular}{ccc}
            \includegraphics[width=0.28\textwidth]{./figures/combined_B\b T_2.0deg.png} &
            \includegraphics[width=0.28\textwidth]{./figures/combined_B\b T_4.0deg.png} &
            \includegraphics[width=0.28\textwidth]{./figures/combined_B\b T_6.0deg.png} \\
            {\tiny $\theta = 2.0^\circ$} & {\tiny $\theta = 4.0^\circ$} & {\tiny $\theta = 6.0^\circ$} \\[0.3cm]
            \includegraphics[width=0.28\textwidth]{./figures/combined_B\b T_8.0deg.png} &
            \includegraphics[width=0.28\textwidth]{./figures/combined_B\b T_10.0deg.png} & \\
            {\tiny $\theta = 8.0^\circ$} & {\tiny $\theta = 10.0^\circ$} & \\
        \end{tabular}
        \vfill
        \begin{itemize}
            \item {\color{magenta}Magenta}: Deuteron (380 MeV (190MeV/u)) trajectories
            \item {\color{blue}Blue}: Proton trajectories (px = $\pm 100 MeV/c, \pm 150MeV/c$)
        \end{itemize}
    \end{frame}
}

\subsection{detailed scan}\label{detailed}
\begin{frame}
    \frametitle{more detailed config}
% 由于1.2 以下磁场, 氘核从neutron exit window 出来, 大于1.2T 质子容易撞击 所以选择1.0到1.2T来之间进行更详细的参数扫描测试
    For magnetic field settings below 1.2 T, deuterons exit from the neutron exit window. Above 1.2 T, protons are prone to collisions. Therefore, a more detailed parameter scan test is conducted between 1.0 and 1.2 T.

\end{frame}

% /home/tian/workspace/dpol/smsimulator5.5/docs/reports/dpolconfig/figures/detailMag

\foreach \detailb in {100,105,110,115,120} {
    \begin{frame}{Combined Trajectories:$B = \fpeval{\detailb * 0.01}$ T}
        \centering
        \begin{tabular}{ccc}
            \includegraphics[width=0.28\textwidth]{figures/detailMag/combined_B\detailb T_0.0deg.png} &
            \includegraphics[width=0.28\textwidth]{figures/detailMag/combined_B\detailb T_1.0deg.png} &
            \includegraphics[width=0.28\textwidth]{figures/detailMag/combined_B\detailb T_2.0deg.png} \\
            {\tiny $\theta = 0.0^\circ$} & {\tiny $\theta = 1.0^\circ$} & {\tiny $\theta = 2.0^\circ$} \\[0.3cm]
            \includegraphics[width=0.28\textwidth]{figures/detailMag/combined_B\detailb T_3.0deg.png} &
            \includegraphics[width=0.28\textwidth]{figures/detailMag/combined_B\detailb T_4.0deg.png} & \\
            {\tiny $\theta = 3.0^\circ$} & {\tiny $\theta = 4.0^\circ$} & \\
        \end{tabular}
        \vfill
        \color{magenta}Magenta Deuteron (380 MeV), \color{yellow}yellow: Proton  trajectories \color{green}green: neutron
        
       \end{frame}
    
    \begin{frame}{Combined Trajectories:$B = \fpeval{\detailb * 0.01}$ T}
                \centering
        \begin{tabular}{ccc}
            \includegraphics[width=0.28\textwidth]{figures/detailMag/combined_B\detailb T_5.0deg.png} &
            \includegraphics[width=0.28\textwidth]{figures/detailMag/combined_B\detailb T_6.0deg.png} &
            \includegraphics[width=0.28\textwidth]{figures/detailMag/combined_B\detailb T_7.0deg.png} \\
            {\tiny $\theta = 5.0^\circ$} & {\tiny $\theta = 6.0^\circ$} & {\tiny $\theta = 7.0^\circ$} \\[0.3cm]
            \includegraphics[width=0.28\textwidth]{figures/detailMag/combined_B\detailb T_8.0deg.png} &
            \includegraphics[width=0.28\textwidth]{figures/detailMag/combined_B\detailb T_9.0deg.png} & \\
            {\tiny $\theta = 8.0^\circ$} & {\tiny $\theta = 9.0^\circ$} & \\
        \end{tabular}
        \vfill
        \color{magenta}Magenta Deuteron (380 MeV), \color{yellow}yellow: Proton  trajectories \color{green}green: neutron
    \end{frame}
}


\begin{frame}
    \frametitle{final choice of config}
    1.15T , 3deg
\begin{columns}
    \begin{column}{0.5\textwidth}
\begin{figure}[ht]
	\centering
	\includegraphics[width=0.4\linewidth]{figures/2026-02-05-21-11-01.png}
	\caption{1.15T , 3deg configuration visualization}
	\label{fig:2026-02-05-21-11-01}
\end{figure}
    \end{column}
    \begin{column}{0.5\textwidth}
\begin{figure}[ht]
	\centering
	\includegraphics[width=0.4\linewidth]{figures/2026-02-05-21-15-17.png}
	\caption{Caption}
	\label{fig:2026-02-05-21-15-17}
\end{figure}
    \end{column}
\end{columns} 



\end{frame}
\subsection{target outside magnetic}

\begin{frame}
    \frametitle{Rationale for Target Placement}
    
    Why the target must be placed inside the magnetic field:
    
    \vspace{0.5cm}
    
    \begin{itemize}
        \item \textbf{Strong $B$-field Case:} 
        \begin{itemize}
            \item Difficult proton track reconstruction.
        \end{itemize}
        
        \item \textbf{Weak $B$-field Case:} 
        \begin{itemize}
            \item Risk of protons hitting the exit window.
        \end{itemize}
        
        \item \textbf{General Constraint:} 
        \begin{itemize}
            \item Poor neutron geometric acceptance.
        \end{itemize}
    \end{itemize}
\end{frame}

% [EN] Target at 0 deg configuration / [CN] 0度靶点配置
% figures: combined_B080T_0deg.png ... combined_B200T_0deg.png

\begin{frame}{Target Outside Magnet: 0 deg Configuration}
    \begin{figure}[htbp]
        \centering
        % Row 1: 080T, 100T, 120T, 140T
        \begin{tabular}{cccc}
            \includegraphics[width=0.22\textwidth]{figures/combined_B080T_0deg.png} &
            \includegraphics[width=0.22\textwidth]{figures/combined_B100T_0deg.png} &
            \includegraphics[width=0.22\textwidth]{figures/combined_B120T_0deg.png} &
            \includegraphics[width=0.22\textwidth]{figures/combined_B140T_0deg.png} \\
            {\tiny 0.80 T} & {\tiny 1.00 T} & {\tiny 1.20 T} & {\tiny 1.40 T} \\[0.2cm]
            % Row 2: 160T, 180T, 200T
            \includegraphics[width=0.22\textwidth]{figures/combined_B160T_0deg.png} &
            \includegraphics[width=0.22\textwidth]{figures/combined_B180T_0deg.png} &
            \includegraphics[width=0.22\textwidth]{figures/combined_B200T_0deg.png} & \\
            {\tiny 1.60 T} & {\tiny 1.80 T} & {\tiny 2.00 T} & \\
        \end{tabular}
        \caption{Target at (0, 0, -4m), Angle = 0 deg}
    \end{figure}
    \vfill
    \begin{itemize}
        \item {\color{magenta}Magenta}: Deuteron (380 MeV)
        \item {\color{blue}Blue}: Proton + Neutron (Px = $\pm$100, $\pm$150 MeV/c)
    \end{itemize}
\end{frame}



\begin{frame}
\begin{figure}[!ht] % 1. 使用 !ht 强制当前位置
    \centering
    
    % --- 第一行 ---
    \begin{subfigure}[b]{0.4\textwidth} % 稍微调大一点宽度利用空间,或者保持 0.4
        \centering
        % 2. 加入 trim 和 clip 裁剪白边 (数值需根据你的原图微调)
        \includegraphics[width=\textwidth, trim=30 30 30 30, clip]{figures/config_0.80T_0deg.png}
        \caption{0.80 T}
        \label{fig:0.80T}
    \end{subfigure}
    \hfill
    \begin{subfigure}[b]{0.4\textwidth}
        \centering
        \includegraphics[width=\textwidth, trim=30 30 30 30, clip]{figures/config_1.00T_0deg.png}
        \caption{1.00 T}
        \label{fig:1.00T}
    \end{subfigure}
    
    % --- 压缩行间距 ---
    \vspace{-0.2cm} % 3. 如果两行中间太宽,用负值往上拉
    
    % --- 第二行 ---
    \begin{subfigure}[b]{0.4\textwidth}
        \centering
        \includegraphics[width=\textwidth, trim=30 30 30 30, clip]{figures/config_1.20T_0deg.png}
        \caption{1.20 T}
        \label{fig:1.20T}
    \end{subfigure}
    \hfill
    \begin{subfigure}[b]{0.4\textwidth}
        \centering
        \includegraphics[width=\textwidth, trim=30 30 30 30, clip]{figures/config_1.40T_0deg.png}
        \caption{1.40 T}
        \label{fig:1.40T}
    \end{subfigure}
    
    \caption{Configuration of the detector under different magnetic fields.}
    \label{fig:four_configs}
\end{figure}
\end{frame}



% ==========================================
% SECTION: 0 Degrees Configuration
% ==========================================

% --- Frame 1: 0deg (0.80T - 1.40T) ---
\begin{frame}{PDC Position: 0 deg (Low Field)}
    \begin{figure}[!ht]
        \centering
        % Row 1
        \begin{subfigure}[b]{0.4\textwidth}
            \centering
            \includegraphics[width=\textwidth, trim=30 30 30 30, clip]{figures/test_pdc_position/config_0.80T_0deg.png}
            \caption{0.80 T}
        \end{subfigure}
        \hfill
        \begin{subfigure}[b]{0.4\textwidth}
            \centering
            \includegraphics[width=\textwidth, trim=30 30 30 30, clip]{figures/test_pdc_position/config_1.00T_0deg.png}
            \caption{1.00 T}
        \end{subfigure}
        
        \vspace{-0.2cm} 
        
        % Row 2
        \begin{subfigure}[b]{0.4\textwidth}
            \centering
            \includegraphics[width=\textwidth, trim=30 30 30 30, clip]{figures/test_pdc_position/config_1.20T_0deg.png}
            \caption{1.20 T}
        \end{subfigure}
        \hfill
        \begin{subfigure}[b]{0.4\textwidth}
            \centering
            \includegraphics[width=\textwidth, trim=30 30 30 30, clip]{figures/test_pdc_position/config_1.40T_0deg.png}
            \caption{1.40 T}
        \end{subfigure}
        \caption{Configuration at 0 deg: 0.80 -- 1.40 T}
    \end{figure}
\end{frame}

% --- Frame 2: 0deg (1.60T - 2.20T) ---
\begin{frame}{PDC Position: 0 deg (Mid Field)}
    \begin{figure}[!ht]
        \centering
        % Row 1
        \begin{subfigure}[b]{0.4\textwidth}
            \centering
            \includegraphics[width=\textwidth, trim=30 30 30 30, clip]{figures/test_pdc_position/config_1.60T_0deg.png}
            \caption{1.60 T}
        \end{subfigure}
        \hfill
        \begin{subfigure}[b]{0.4\textwidth}
            \centering
            \includegraphics[width=\textwidth, trim=30 30 30 30, clip]{figures/test_pdc_position/config_1.80T_0deg.png}
            \caption{1.80 T}
        \end{subfigure}
        
        \vspace{-0.2cm} 
        
        % Row 2
        \begin{subfigure}[b]{0.4\textwidth}
            \centering
            \includegraphics[width=\textwidth, trim=30 30 30 30, clip]{figures/test_pdc_position/config_2.00T_0deg.png}
            \caption{2.00 T}
        \end{subfigure}
        \hfill
        \begin{subfigure}[b]{0.4\textwidth}
            \centering
            \includegraphics[width=\textwidth, trim=30 30 30 30, clip]{figures/test_pdc_position/config_2.20T_0deg.png}
            \caption{2.20 T}
        \end{subfigure}
        \caption{Configuration at 0 deg: 1.60 -- 2.20 T}
    \end{figure}
\end{frame}

% --- Frame 3: 0deg (2.40T - 3.00T) ---
\begin{frame}{PDC Position: 0 deg (High Field)}
    \begin{figure}[!ht]
        \centering
        % Row 1
        \begin{subfigure}[b]{0.4\textwidth}
            \centering
            \includegraphics[width=\textwidth, trim=30 30 30 30, clip]{figures/test_pdc_position/config_2.40T_0deg.png}
            \caption{2.40 T}
        \end{subfigure}
        \hfill
        \begin{subfigure}[b]{0.4\textwidth}
            \centering
            \includegraphics[width=\textwidth, trim=30 30 30 30, clip]{figures/test_pdc_position/config_2.60T_0deg.png}
            \caption{2.60 T}
        \end{subfigure}
        
        \vspace{-0.2cm} 
        
        % Row 2
        \begin{subfigure}[b]{0.4\textwidth}
            \centering
            \includegraphics[width=\textwidth, trim=30 30 30 30, clip]{figures/test_pdc_position/config_2.80T_0deg.png}
            \caption{2.80 T}
        \end{subfigure}
        \hfill
        \begin{subfigure}[b]{0.4\textwidth}
            \centering
            \includegraphics[width=\textwidth, trim=30 30 30 30, clip]{figures/test_pdc_position/config_3.00T_0deg.png}
            \caption{3.00 T}
        \end{subfigure}
        \caption{Configuration at 0 deg: 2.40 -- 3.00 T}
    \end{figure}
\end{frame}

% ==========================================
% SECTION: 5 Degrees Configuration
% ==========================================

% --- Frame 4: 5deg (0.80T - 1.40T) ---
\begin{frame}{PDC Position: 5 deg (Low Field)}
    \begin{figure}[!ht]
        \centering
        \begin{subfigure}[b]{0.4\textwidth}
            \centering
            \includegraphics[width=\textwidth, trim=30 30 30 30, clip]{figures/test_pdc_position/config_0.80T_5deg.png}
            \caption{0.80 T}
        \end{subfigure}
        \hfill
        \begin{subfigure}[b]{0.4\textwidth}
            \centering
            \includegraphics[width=\textwidth, trim=30 30 30 30, clip]{figures/test_pdc_position/config_1.00T_5deg.png}
            \caption{1.00 T}
        \end{subfigure}
        \vspace{-0.2cm} 
        \begin{subfigure}[b]{0.4\textwidth}
            \centering
            \includegraphics[width=\textwidth, trim=30 30 30 30, clip]{figures/test_pdc_position/config_1.20T_5deg.png}
            \caption{1.20 T}
        \end{subfigure}
        \hfill
        \begin{subfigure}[b]{0.4\textwidth}
            \centering
            \includegraphics[width=\textwidth, trim=30 30 30 30, clip]{figures/test_pdc_position/config_1.40T_5deg.png}
            \caption{1.40 T}
        \end{subfigure}
        \caption{Configuration at 5 deg: 0.80 -- 1.40 T}
    \end{figure}
\end{frame}

% --- Frame 5: 5deg (1.60T - 2.20T) ---
\begin{frame}{PDC Position: 5 deg (Mid Field)}
    \begin{figure}[!ht]
        \centering
        \begin{subfigure}[b]{0.4\textwidth}
            \centering
            \includegraphics[width=\textwidth, trim=30 30 30 30, clip]{figures/test_pdc_position/config_1.60T_5deg.png}
            \caption{1.60 T}
        \end{subfigure}
        \hfill
        \begin{subfigure}[b]{0.4\textwidth}
            \centering
            \includegraphics[width=\textwidth, trim=30 30 30 30, clip]{figures/test_pdc_position/config_1.80T_5deg.png}
            \caption{1.80 T}
        \end{subfigure}
        \vspace{-0.2cm} 
        \begin{subfigure}[b]{0.4\textwidth}
            \centering
            \includegraphics[width=\textwidth, trim=30 30 30 30, clip]{figures/test_pdc_position/config_2.00T_5deg.png}
            \caption{2.00 T}
        \end{subfigure}
        \hfill
        \begin{subfigure}[b]{0.4\textwidth}
            \centering
            \includegraphics[width=\textwidth, trim=30 30 30 30, clip]{figures/test_pdc_position/config_2.20T_5deg.png}
            \caption{2.20 T}
        \end{subfigure}
        \caption{Configuration at 5 deg: 1.60 -- 2.20 T}
    \end{figure}
\end{frame}

% --- Frame 6: 5deg (2.40T - 3.00T) ---
\begin{frame}{PDC Position: 5 deg (High Field)}
    \begin{figure}[!ht]
        \centering
        \begin{subfigure}[b]{0.4\textwidth}
            \centering
            \includegraphics[width=\textwidth, trim=30 30 30 30, clip]{figures/test_pdc_position/config_2.40T_5deg.png}
            \caption{2.40 T}
        \end{subfigure}
        \hfill
        \begin{subfigure}[b]{0.4\textwidth}
            \centering
            \includegraphics[width=\textwidth, trim=30 30 30 30, clip]{figures/test_pdc_position/config_2.60T_5deg.png}
            \caption{2.60 T}
        \end{subfigure}
        \vspace{-0.2cm} 
        \begin{subfigure}[b]{0.4\textwidth}
            \centering
            \includegraphics[width=\textwidth, trim=30 30 30 30, clip]{figures/test_pdc_position/config_2.80T_5deg.png}
            \caption{2.80 T}
        \end{subfigure}
        \hfill
        \begin{subfigure}[b]{0.4\textwidth}
            \centering
            \includegraphics[width=\textwidth, trim=30 30 30 30, clip]{figures/test_pdc_position/config_3.00T_5deg.png}
            \caption{3.00 T}
        \end{subfigure}
        \caption{Configuration at 5 deg: 2.40 -- 3.00 T}
    \end{figure}
\end{frame}

% ==========================================
% SECTION: 10 Degrees Configuration
% ==========================================

% --- Frame 7: 10deg (0.80T - 1.40T) ---
\begin{frame}{PDC Position: 10 deg (Low Field)}
    \begin{figure}[!ht]
        \centering
        \begin{subfigure}[b]{0.4\textwidth}
            \centering
            \includegraphics[width=\textwidth, trim=30 30 30 30, clip]{figures/test_pdc_position/config_0.80T_10deg.png}
            \caption{0.80 T}
        \end{subfigure}
        \hfill
        \begin{subfigure}[b]{0.4\textwidth}
            \centering
            \includegraphics[width=\textwidth, trim=30 30 30 30, clip]{figures/test_pdc_position/config_1.00T_10deg.png}
            \caption{1.00 T}
        \end{subfigure}
        \vspace{-0.2cm} 
        \begin{subfigure}[b]{0.4\textwidth}
            \centering
            \includegraphics[width=\textwidth, trim=30 30 30 30, clip]{figures/test_pdc_position/config_1.20T_10deg.png}
            \caption{1.20 T}
        \end{subfigure}
        \hfill
        \begin{subfigure}[b]{0.4\textwidth}
            \centering
            \includegraphics[width=\textwidth, trim=30 30 30 30, clip]{figures/test_pdc_position/config_1.40T_10deg.png}
            \caption{1.40 T}
        \end{subfigure}
        \caption{Configuration at 10 deg: 0.80 -- 1.40 T}
    \end{figure}
\end{frame}

% --- Frame 8: 10deg (1.60T - 2.20T) ---
\begin{frame}{PDC Position: 10 deg (Mid Field)}
    \begin{figure}[!ht]
        \centering
        \begin{subfigure}[b]{0.4\textwidth}
            \centering
            \includegraphics[width=\textwidth, trim=30 30 30 30, clip]{figures/test_pdc_position/config_1.60T_10deg.png}
            \caption{1.60 T}
        \end{subfigure}
        \hfill
        \begin{subfigure}[b]{0.4\textwidth}
            \centering
            \includegraphics[width=\textwidth, trim=30 30 30 30, clip]{figures/test_pdc_position/config_1.80T_10deg.png}
            \caption{1.80 T}
        \end{subfigure}
        \vspace{-0.2cm} 
        \begin{subfigure}[b]{0.4\textwidth}
            \centering
            \includegraphics[width=\textwidth, trim=30 30 30 30, clip]{figures/test_pdc_position/config_2.00T_10deg.png}
            \caption{2.00 T}
        \end{subfigure}
        \hfill
        \begin{subfigure}[b]{0.4\textwidth}
            \centering
            \includegraphics[width=\textwidth, trim=30 30 30 30, clip]{figures/test_pdc_position/config_2.20T_10deg.png}
            \caption{2.20 T}
        \end{subfigure}
        \caption{Configuration at 10 deg: 1.60 -- 2.20 T}
    \end{figure}
\end{frame}

% --- Frame 9: 10deg (2.40T - 3.00T) ---
\begin{frame}{PDC Position: 10 deg (High Field)}
    \begin{figure}[!ht]
        \centering
        \begin{subfigure}[b]{0.4\textwidth}
            \centering
            \includegraphics[width=\textwidth, trim=30 30 30 30, clip]{figures/test_pdc_position/config_2.40T_10deg.png}
            \caption{2.40 T}
        \end{subfigure}
        \hfill
        \begin{subfigure}[b]{0.4\textwidth}
            \centering
            \includegraphics[width=\textwidth, trim=30 30 30 30, clip]{figures/test_pdc_position/config_2.60T_10deg.png}
            \caption{2.60 T}
        \end{subfigure}
        \vspace{-0.2cm} 
        \begin{subfigure}[b]{0.4\textwidth}
            \centering
            \includegraphics[width=\textwidth, trim=30 30 30 30, clip]{figures/test_pdc_position/config_2.80T_10deg.png}
            \caption{2.80 T}
        \end{subfigure}
        \hfill
        \begin{subfigure}[b]{0.4\textwidth}
            \centering
            \includegraphics[width=\textwidth, trim=30 30 30 30, clip]{figures/test_pdc_position/config_3.00T_10deg.png}
            \caption{3.00 T}
        \end{subfigure}
        \caption{Configuration at 10 deg: 2.40 -- 3.00 T}
    \end{figure}
\end{frame}


\section{filter}
 
\subsection{nebula acceptance}

\begin{frame}
    

The neutron acceptance depends on the beam bending angle. Although different magnetic field settings cause slight differences in position, the impact of the magnetic field is negligible. The angle should be smaller than $10^{\circ}$. Furthermore, the neutron distribution does not correlate with the proton changes; for neutrons within a specific region, the corresponding protons are distributed across nearly the entire space, indicating no significant correlation.

% 某一个区域内中子, 对应的质子也几乎弥散在全部的空间上 ,没有明显的相关性. (翻译此句话)

For neutrons within a specific region, the corresponding protons are distributed across nearly the entire space, indicating no significant correlation.
\end{frame}






\end{document}