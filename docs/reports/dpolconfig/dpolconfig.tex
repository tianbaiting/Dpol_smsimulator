
\documentclass[aspectratio=169]{beamer}
\usetheme{Madrid}
\usecolortheme{default}

% Packages
\usepackage{graphicx}
\usepackage{amsmath}
\usepackage{amssymb}
\usepackage{booktabs}
\usepackage{listings}
\usepackage{xcolor}
\usepackage{hyperref}
\usepackage{tikz}
\usepackage{subcaption}
\usepackage{graphicx}
\usetikzlibrary{shapes,arrows,positioning}

% Code listing settings
\lstset{
  basicstyle=\ttfamily\footnotesize,
  breaklines=true,
  frame=single,
  numbers=left,
  numberstyle=\tiny\color{gray},
}
\title{dpol config}
\author{Tian Baiting}
\institute{SAMURAI Collaboration}

% Footer customization
\setbeamertemplate{footline}[frame number]

\begin{document}

% Title slide
\begin{frame}
\titlepage
\end{frame}

\begin{frame}
  %  list of content 生成目录
\tableofcontents  
\end{frame}

\section{config}

\subsection{dpol hit exitwindow}

\begin{frame}
    \frametitle{Operational Constraints} % 或者是 Magnetic Field Requirements
    
    \begin{itemize}
        \item \textbf{Requirement for Exit Window Configuration:}
        A magnetic field of $B \geq 2.0~\text{T}$ is mandatory.
        
        \item \textbf{Rationale:}
        At lower field strengths ($B < 2.0~\text{T}$), the trajectory of beam line risks hitting on the vacuum exit window frame.
    \end{itemize}
\end{frame}

\begin{figure}[ht]
	\centering
	\includegraphics[width=0.8\linewidth]{figures/2026-01-18-14-46-58.png}
	\caption{2.0T}
	\label{fig:2026-01-18-14-46-58}
\end{figure}

\begin{figure}[ht]
	\centering
	\includegraphics[width=0.8\linewidth]{figures/2026-01-18-14-48-03.png}
	\caption{1.8T}
	\label{fig:2026-01-18-14-48-03}
\end{figure}

\begin{figure}[ht]
	\centering
	\includegraphics[width=0.8\linewidth]{figures/2026-01-18-14-48-23.png}
	\caption{1.6T}
	\label{fig:2026-01-18-14-48-23}
\end{figure}

\begin{figure}[ht]
	\centering
	\includegraphics[width=0.8\linewidth]{figures/2026-01-18-15-06-31.png}
	\caption{1.4T}
	\label{fig:2026-01-18-15-06-31}
\end{figure}


\begin{figure}[ht]
	\centering
	\includegraphics[width=0.8\linewidth]{figures/2026-01-18-15-06-55.png}
	\caption{1.2T}
	\label{fig:2026-01-18-15-06-55}
\end{figure}

\begin{figure}[ht]
	\centering
	\includegraphics[width=0.8\linewidth]{figures/2026-01-18-15-07-40.png}
	\caption{1.0T}
	\label{fig:2026-01-18-15-07-40}
\end{figure}


\begin{figure}[ht]
	\centering
	\includegraphics[width=0.8\linewidth]{figures/2026-01-18-15-08-25.png}
	\caption{0.8T}
	\label{fig:2026-01-18-15-08-25}
\end{figure}


\subsection{target outside magnetic}

\begin{frame}
    \frametitle{Rationale for Target Placement}
    
    Why the target must be placed inside the magnetic field:
    
    \vspace{0.5cm}
    
    \begin{itemize}
        \item \textbf{Strong $B$-field Case:} 
        \begin{itemize}
            \item Difficult proton track reconstruction.
        \end{itemize}
        
        \item \textbf{Weak $B$-field Case:} 
        \begin{itemize}
            \item Risk of protons hitting the exit window.
        \end{itemize}
        
        \item \textbf{General Constraint:} 
        \begin{itemize}
            \item Poor neutron geometric acceptance.
        \end{itemize}
    \end{itemize}
\end{frame}

\begin{frame}
\begin{figure}[!ht] % 1. 使用 !ht 强制当前位置
    \centering
    
    % --- 第一行 ---
    \begin{subfigure}[b]{0.4\textwidth} % 稍微调大一点宽度利用空间,或者保持 0.4
        \centering
        % 2. 加入 trim 和 clip 裁剪白边 (数值需根据你的原图微调)
        \includegraphics[width=\textwidth, trim=30 30 30 30, clip]{figures/config_0.80T_0deg.png}
        \caption{0.80 T}
        \label{fig:0.80T}
    \end{subfigure}
    \hfill
    \begin{subfigure}[b]{0.4\textwidth}
        \centering
        \includegraphics[width=\textwidth, trim=30 30 30 30, clip]{figures/config_1.00T_0deg.png}
        \caption{1.00 T}
        \label{fig:1.00T}
    \end{subfigure}
    
    % --- 压缩行间距 ---
    \vspace{-0.2cm} % 3. 如果两行中间太宽,用负值往上拉
    
    % --- 第二行 ---
    \begin{subfigure}[b]{0.4\textwidth}
        \centering
        \includegraphics[width=\textwidth, trim=30 30 30 30, clip]{figures/config_1.20T_0deg.png}
        \caption{1.20 T}
        \label{fig:1.20T}
    \end{subfigure}
    \hfill
    \begin{subfigure}[b]{0.4\textwidth}
        \centering
        \includegraphics[width=\textwidth, trim=30 30 30 30, clip]{figures/config_1.40T_0deg.png}
        \caption{1.40 T}
        \label{fig:1.40T}
    \end{subfigure}
    
    \caption{Configuration of the detector under different magnetic fields.}
    \label{fig:four_configs}
\end{figure}
\end{frame}


\section{filter}
 
\subsection{pdc position}


\subsection{pdc acceptence}


\subsection{}


\end{document}